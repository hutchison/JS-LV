\begin{longtable}{rXp{5cm}p{4cm}}
   & \centering\textbf{Titel} & \centering\textbf{Dozent} &
  \textbf{Institut/Lehrstuhl}\\
  1. & Eröffnung und Grußworte & Prof. Wolfgang Schareck & Rektor der
  Universität Rostock\\

  2. & Spielend lernen mit Computern & Prof. Alke Martens & Institut für
  Informatik\\

  3. & Ernst und Spiel als Modus von Bildung: vom Buch zum Edutainment & Prof.
  Wolfgang Nieke & Institut für Pädagogik und Sozialpädagogik\\

  4. & Künstliche Intelligenz spielend lernen: Das Spiel Gorge & Prof. Klaus P.
  Jantke & Fraunhofer Institut für Digitale Medientechnologie Ilmenau/Erfurt\\

  5. & Mathematische Modelle des Lernens & Prof. Lars Schwabe & Institut für
  Informatik\\

  6. & Was haben interaktive Medien und eLearning in der Schule mit einer
  Satellitenschüssel in der afrikanischen Savanne zu tun? & Dipl.-Päd. Jan
  Hartmann & Institut für Qualitätsentwicklung Mecklenburg-Vorpommern\\

  7. & Spielend lernen in der Alphabetisierung & Dipl.-Päd. Steffen Malo &
  Fraunhofer Institut für Graphische Datenverarbeitung\\

  8. & eLearning Didaktik & Dr.-Ing. Sybille Hambach & Baltic College\\

  9. & Immersive Learning: Systematische Verschmelzung von virtueller und
  Präsenzlehre & Dipl.-Inf. Raphael Zender & Institut für Infomatik\\

  10. & Neue Kompetenzen - Technologie-gestützt spielend erwerben?! &
  Prof. Dietrich Albert & Universität Graz,

  Knowledge Management Institute\\

  11. & ePortfolios – Vom „Hammer-sucht-Nagel-Spiel“ zur didaktischen Reflexion
      & Prof. Thomas Hugo Häcker & Institut für Schulpädagogik\\

  12. & Kann man in Blogs und Wikis auch etwas lernen? & Prof. Dr. Clemens Cap &
  Institut für Informatik\\

  13. & Jesus im Adventure Game: Bibeldidatik auf neuen Wegen & Prof. Martin
  Rösel & Theologische Fakultät,

  Fachgebiet Altes Testament\\

  14. & Lernen in 3D-Welten & Dr. Markus Walber\,/\,Dennis Schäffer &
  Universität Bielefeld,

  Fakultät für Erziehungswissenschaft\\
\end{longtable}
