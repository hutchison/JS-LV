\documentclass[%
a4paper, %apaper,   % alle weiteren Papierformat einstellbar
%empty,             % keine Seitenzahlen
11pt,               % Schriftgröße (12pt, 11pt (Standard))
leqno,              % Nummerierung von Gleichungen links
fleqn,              % Ausgabe von Gleichungen linksbündig
%smallheadings,
%abstracton,
]
{scrartcl}
%{memoir}

%% Deutsche Anpassungen
\usepackage[ngerman]{babel}

\usepackage{xltxtra, fontspec, xunicode}

\newcommand{\headerfont}{Gill Sans Std}

% TODO: Die Schrift auf eine ändern, die man auch auf dem System findet.
%\setmainfont[Mapping=tex-text]{Palatino LT Std}
\setmainfont[Mapping=tex-text]{Minion Pro}
\setkomafont{sectioning}{\fontspec{\headerfont}}
\setkomafont{section}{\LARGE}
\setkomafont{subsection}{\Large}
\setkomafont{subsubsection}{\large}
\setkomafont{title}{\fontspec{\headerfont}}
%\linespread{1.05}

%obligatorischer Mathekram:
\usepackage{amssymb,amstext}
%\usepackage{stmaryrd}
\usepackage[sumlimits]{amsmath}
%\usepackage{eulervm}
%\usepackage{mnsymbol}

%nützliche Extras:
\usepackage{array,
  %hhline,
  tabularx,
  %enumerate,
  %color,
  %setspace,
  %booktabs,
  %cite,
  %caption,
  %lineno,
  %lastpage,
  %algorithm,
  %algorithmic,
  %listings,
  %bussproofs,
  %ulem,
  %multirow,
  %natbib,
  %url,
  verse,
  ltxtable,
}
\usepackage[
  xetex,
  colorlinks=false,
  linkbordercolor={0 0.5 1},
  %linkcolor=blue,
  %pdfnewwindow=true
]
{hyperref}

\usepackage[
  cm,
  %headings
]{fullpage}
%\usepackage[left=1cm,right=1cm,top=1cm,bottom=1cm,includeheadfoot]{geometry}

%\usepackage{fancyhdr}
%\pagestyle{fancy}
%\fancyhf{}

%%\fancyhead[C]{Lehrveranstaltungskatalog für das Juniorstudium der Universität Rostock}
%%\fancyhead[C]{}
%%\fancyhead[R]{}

%% \fancyfoot[L]{}
%\fancyfoot[C]{\thepage}
%% \fancyfoot[R]{\thepage / \pageref{LastPage}}

%%% Linie oben/unten
%\renewcommand{\headrulewidth}{0.0pt}
%\renewcommand{\footnoterule}{}

\parindent 0pt

%% Pakete für Grafiken & Abbildungen
%\usepackage{graphicx}
%\usepackage{tikz}      %%TeX ist kein Zeichenprogramm
%\usetikzlibrary{calc,arrows}

%% Lochmarken; auskommentieren, falls unerwünscht
%\usepackage{eso-pic}
%\AddToShipoutPicture*{
%    \put(0,68.5mm){\rule{15pt}{0.5pt}}
%    \put(0,0.5\paperheight){\rule{20pt}{0.5pt}}
%    \put(0,228.5mm){\rule{15pt}{0.5pt}}
%}

%\usepackage{pdflscape}

\newcommand{\code}[1]{{\fontspec[Mapping=tex-text]{Bitstream Vera Sans Mono}\footnotesize#1}}
\newcommand{\tinycode}[1]{{\fontspec[Mapping=tex-text]{Bitstream Vera Sans Mono}\tiny#1}}

\newcommand{\attrib}[1]{\nopagebreak{\raggedleft\small #1\par}}

\title{Lehrveranstaltungskatalog für das Juniorstudium der Universität Rostock}
\author{\today}
\date{}

\renewcommand{\thesection}{\space}
\renewcommand{\thesubsection}{\arabic{subsection}}
\renewcommand{\thesubsubsection}{\arabic{subsection}.\arabic{subsubsection}}
\setcounter{tocdepth}{2}

\begin{document}

\maketitle

\begin{abstract}
  \begin{description}
    \item[Wichtig:] Die Teilnahme an den Vorlesungen hat keinerlei Einfluss auf
      das Zulassungsverfahren für NC-beschränkte Studienfächer an der
      Universität Rostock!
  \end{description}
\end{abstract}

\tableofcontents

\begin{quote}
  \begin{description}
    \item[Vorbemerkung:] SWS steht für \textit{Semesterwochenstunde}. Eine SWS
      entspricht 45 Minuten, beinhaltet jedoch nicht die Vor- und
      Nachbereitungszeit.
  \end{description}
\end{quote}

\newpage

\section{Übersicht} % (fold)
\label{sec:Übersicht}

\subsection{Informatik} % (fold)
\label{sec:Informatik}

\renewcommand{\arraystretch}{1.5}

\begin{tabularx}{\textwidth}{cXcp{5.6cm}}
  \textbf{Nr.} & \textbf{Veranstaltung} & \textbf{SWS} &
  \multicolumn{1}{c}{\textbf{Dozent}}\\
  \hline
  1.1 & \hyperref[ssub:Funktionale Programmierung]{Funktionale Programmierung} &
    3 & Prof.\,Dr. Thomas Kirste\\
  1.2 & \hyperref[ssub:Grundlagen der Technischen Informatik]{Grundlagen der
  Technischen Informatik} & 3 & Prof. Dr. Peter Luksch\\
  1.3 & \hyperref[ssub:Einführung in die Programmierung mit C]{Einführung in die
  Programmierung mit C} & 2 & Lutz Hellmig\\
  1.4 & \hyperref[ssub:Abstrakte Datentypen]{Abstrakte Datentypen} & 2 &
  Prof.\,Dr. Alke Martens\\
  1.5 & \hyperref[ssub:Algorithmen und Datenstrukturen]{Algorithmen und
Datenstrukturen} & 2 & Prof.\,Dr. Thomas Kirste \\
  1.6 & \hyperref[ssub:Einführung in die Informatik für BWLer und
VWLer]{Einführung Informatik für BWLer, VWLer} & 2 & Dr. Meike Klettke\\
  1.7 & \hyperref[ssub:Einführung in die Praktische Informatik]{Einführung in
die Praktische Informatik} & 3 & Prof.\,Dr.-Ing.\,habil.  Ralf Salomon\\
  1.8 & \hyperref[ssub:Logik]{Logik} & 2 &
  Prof.\,Dr.\,rer.\,nat.\,habil. Karsten Wolf\\
\end{tabularx}

% subsection Informatik (end)

\subsection{Chemie/Medizin} % (fold)
\label{sec:Chemie/Medizin}

\begin{tabularx}{\textwidth}{cXcp{5.6cm}}
  \textbf{Nr.} & \textbf{Veranstaltung} & \textbf{SWS} &
  \multicolumn{1}{c}{\textbf{Dozent}}\\
  \hline
  2.1 & \hyperref[ssub:Medizinische Neurobiologie]{Medizinische Neurobiologie} &
    2 & PD.\,Dr. Christian Andressen\\
  2.2 & \hyperref[ssub:Anatomie des Nervensystems]{Anatomie des Nervensystems} &
  2,7 & Prof. Wree\\
  2.3 & \hyperref[ssub:Anatomie der Sinnesorgane]{Anatomie der Sinnesorgane} & 1
      & PD.\,Dr. Christian Andressen\\
  2.4 & \hyperref[ssub:Grundlagen Chemie für Mediziner]{Grundlagen Chemie für
Mediziner}

  (Vorlesung \& Experimente) & 2 +1 & Dr.
  Gisela Boeck\\
  2.5 & \hyperref[ssub:Technische Chemie]{Technische Chemie} & 2 & Prof. Udo
  Kragl\\
  2.6 & \hyperref[ssub:Anatomie für Nichtmediziner -- Teil I]{Anatomie für
  Nichtmediziner -- Teil I} & 1 & Prof.\,Dr. Andreas Wree\\
  2.7 & \hyperref[ssub:Allg. Zytologie und Histologie]{Allg. Zytologie und
  Histologie} & 2 & PD.\,Dr. Oliver Schmitt\\
\end{tabularx}

% subsection Chemie/Medizin (end)

\subsection{Biologie} % (fold)
\label{sec:Biologie}

\begin{tabularx}{\textwidth}{cXcp{5.6cm}}
  \textbf{Nr.} & \textbf{Veranstaltung} & \textbf{SWS} &
  \multicolumn{1}{c}{\textbf{Dozent}}\\
  \hline
  3.1 & \hyperref[ssub:Grundlagen der Genetik]{Grundlagen der Genetik} & 2 &
  Prof.\,Dr. Reinhard Schröder\\
  3.2 & \hyperref[ssub:Evolution]{Evolution} & 2 & Prof.\,Dr. Dieter Weiss\\
\end{tabularx}

% subsection Biologie (end)

\subsection{Anglistik} % (fold)
\label{sec:Anglistik}

\begin{tabularx}{\textwidth}{cXcp{5.6cm}}
  \textbf{Nr.} & \textbf{Veranstaltung} & \textbf{SWS} &
  \multicolumn{1}{c}{\textbf{Dozent}}\\
  \hline
  4.1 & \hyperref[ssub:7th Fulbright Lecture Series: American Culture, Past and
  Present]{7th Fulbright Lecture Series: American Culture, Past and Present} & 2
  & Prof.\,Dr. Gabriele Linke\\
\end{tabularx}

% subsection Anglistik (end)

\subsection{Geschichte} % (fold)
\label{sec:Geschichte}

\begin{tabularx}{\textwidth}{cXcp{5.6cm}}
  \textbf{Nr.} & \textbf{Veranstaltung} & \textbf{SWS} &
  \multicolumn{1}{c}{\textbf{Dozent}}\\
  \hline
  5.1 & \hyperref[ssub:Epochen im Überblick: Von der Reformation zum
  Westfälischen Frieden]{Epochen im Überblick: Von der Reformation zum
  Westfälischen Frieden} & 2 & Prof. Kersten Krüger\\
  5.2 & \hyperref[ssub:Epochen im Überblick: Von der Reformation zum
  Westfälischen Frieden (Teil 2)]{Epochen im Überblick: Von der Reformation zum
  Westfälischen Frieden (Teil 2)} & 2 & Prof. Kersten Krüger\\
  5.3 & \hyperref[ssub:Europäischer Absolutismus 1648--1789. Teil
  1]{Europäischer Absolutismus 1648--1789. Teil 1} & 2 & Prof. Kersten Krüger\\
  5.4 & \hyperref[ssub:Europäischer Absolutismus 1648--1789. Teil
  2]{Europäischer Absolutismus 1648--1789. Teil 2} & 2 & Prof. Kersten Krüger\\
  5.5 & \hyperref[ssub:Diaspora]{Diaspora} & 2 & Prof.\,Dr. Peter Burschel\\
  5.6 & \hyperref[ssub:Die Stadt der Moderne (16.--20. Jahrhundert)]{Die Stadt
  der Moderne (16.--20. Jahrhundert)} & 2 & Prof. Kersten Krüger\\
  5.7 & \hyperref[ssub:Geschichte des deutschen Kommunismus (Teil 1:
  1918--1945)]{Geschichte des deutschen Kommunismus (Teil 1: 1918--1945)} & 2 &
  Prof.\,Dr. Werner Müller\\
\end{tabularx}

% subsection Geschichte (end)

\subsection{Theologie} % (fold)
\label{sec:Theologie}

\begin{tabularx}{\textwidth}{cXcp{5.6cm}}
  \textbf{Nr.} & \textbf{Veranstaltung} & \textbf{SWS} &
  \multicolumn{1}{c}{\textbf{Dozent}}\\
  \hline
  6.1 & \hyperref[ssub:Einführung in die Geschichte des Christentums]{Einführung
  in die Geschichte des Christentums} & 2 & Prof. Heinrich Holze\\
  6.2 & \hyperref[ssub:Bibelkunde des Neuen Testaments]{Bibelkunde des Neuen
  Testaments} & 2 & Dr. Klaus Bull\\
  6.3 & \hyperref[ssub:Geschichte der frühjüdischen Literatur]{Geschichte der
  frühjüdischen Literatur} & 2 & Dr. Klaus Bull\\
\end{tabularx}

% subsection Theologie (end)

\subsection{Germanistik/Kommunikation} % (fold)
\label{sec:Germanistik/Kommunikation}

\begin{tabularx}{\textwidth}{cXcp{5.6cm}}
  \textbf{Nr.} & \textbf{Veranstaltung} & \textbf{SWS} &
  \multicolumn{1}{c}{\textbf{Dozent}}\\
  \hline
  7.1 & \hyperref[ssub:„Mach mal den Flyer! Du hast das doch studiert!“
  Kommunikationswissenschaft zwischen Klischee und Wirklichkeit]{„Mach mal den
  Flyer! Du hast das doch studiert!“ Kommunikationswissenschaft zwischen
  Klischee und Wirklichkeit} & 2 & Mario Donick, M.\,A.\\
  7.2 & \hyperref[ssub:Sprache -- Medien -- Kommunikation: Von den Anfängen der
  deutschen Sprache zur Internet-Kommunikation]{Sprache -- Medien --
  Kommunikation: Von den Anfängen der deutschen Sprache zur
Internet-Kommunikation} & 2 & Mario Donick, M.\,A.\\
  7.3 & \hyperref[ssub:Stationen deutscher Lyrik]{Stationen deutscher Lyrik} & 2
  & Prof.\,Dr. Holger Helbig\\
\end{tabularx}

% subsection Germanistik/Kommunikation (end)

\subsection{Allgemeine Studien} % (fold)
\label{sec:Allgemeine Studien}

\begin{tabularx}{\textwidth}{cp{14em}cX}
  \textbf{Nr.} & \textbf{Veranstaltung} & \textbf{SWS} &
  \multicolumn{1}{c}{\textbf{Dozent}}\\
  \hline
  8.1 & \hyperref[ssub:Was Bilder (un)sichtbar macht]{Was Bilder (un)sichtbar
  macht} & 2 & interdisziplinär, Organisation: Dr. Christine Bräuning\\
  8.2 & \hyperref[ssub:Leben-Licht-Materie]{Leben-Licht-Materie} & 2 &
  interdisziplinär, Organisation: Dr. Christine Bräuning\\
  8.3 & \hyperref[ssub:Strukturen und Symmetrien]{Strukturen und Symmetrien} & 2
      & interdisziplinär, Organisation: Dr. Christine Bräuning\\
  8.4 & \hyperref[ssub:Spielend Lernen]{Spielend Lernen} & 2 & interdisziplinär,
  Organisation: Dr. Christine Bräuning\\
  8.5 & \hyperref[ssub:„Erfolgreich Altern“]{„Erfolgreich Altern“} & 2 &
  interdisziplinär, verantwortlich: Prof.\,Dr.-Ing. Thomas Kirste\\
  8.6 & \hyperref[ssub:Einführung in die Demographie I]{Einführung in die
  Demographie I} & 2 & Prof.\,Dr. G. Doblhammer-Reiter, Jun.-Prof.\,Dr.
  Roland Rau\\
\end{tabularx}

% subsection Allgemeine Studien (end)

\subsection{Sozialpsychologie} % (fold)
\label{sec:Sozialpsychologie}

\begin{tabularx}{\textwidth}{cXcp{5.6cm}}
  \textbf{Nr.} & \textbf{Veranstaltung} & \textbf{SWS} &
  \multicolumn{1}{c}{\textbf{Dozent}}\\
  \hline
  9.1 & \hyperref[ssub:Einführung in die Sozialpsychologie für
  Lehramtskandidaten]{Einführung in die Sozialpsychologie für
  Lehramtskandidaten} & 1 & Prof.  Dr. Christoph Perleth\\
  9.2 & \hyperref[ssub:Einführung in die pädagogisch-psychologische Diagnostik
  für Lehramtskandidaten]{Einführung in die pädagogisch-psychologische
  Diagnostik für Lehramtskandidaten} & 1 & Prof.\,Dr. Christoph Perleth\\
\end{tabularx}

% subsection Sozialpsychologie (end)

\subsection{Maschinenbau} % (fold)
\label{sec:Maschinenbau}

\begin{tabularx}{\textwidth}{cXcp{5.6cm}}
  \textbf{Nr.} & \textbf{Veranstaltung} & \textbf{SWS} &
  \multicolumn{1}{c}{\textbf{Dozent}}\\
  \hline
  10.1 & \hyperref[ssub:Fertigungslehre I]{Fertigungslehre I} & 3 &
  Prof.\,Dr.-Ing. Martin Wanner\\
\end{tabularx}

% subsection Maschinenbau (end)

\subsection{Philosophie} % (fold)
\label{sec:Philosophie}

\begin{tabularx}{\textwidth}{cXcp{5.6cm}}
  \textbf{Nr.} & \textbf{Veranstaltung} & \textbf{SWS} &
  \multicolumn{1}{c}{\textbf{Dozent}}\\
  \hline
  11.1 & \hyperref[ssub:Aristoteles und die abendländische
  Philosophie]{Aristoteles und die abendländische Philosophie} & 2 &
  Prof.\,Dr. Wolfgang Bernard\\
\end{tabularx}

% subsection Philosophie (end)

%\subsection{Jura} % (fold)
%\label{sec:Jura}

%\begin{tabularx}{\textwidth}{cXcp{5.6cm}}
  %\textbf{Nr.} & \textbf{Veranstaltung} & \textbf{SWS} &
  %\multicolumn{1}{c}{\textbf{Dozent}}\\
  %\hline
  %12.1 & Einführung in die Rechtswissenschaft & 2 & Prof.\,Dr.\,iur. Jörg
  %Benedict\\
%\end{tabularx}

% subsection Jura (end)

\subsection{Wirtschaftswissenschaften} % (fold)
\label{sec:Wirtschaftswissenschaften}

\begin{tabularx}{\textwidth}{cXcp{5.6cm}}
  \textbf{Nr.} & \textbf{Veranstaltung} & \textbf{SWS} &
  \multicolumn{1}{c}{\textbf{Dozent}}\\
  \hline
  13.1 & \hyperref[ssub:Einführung in die Volkswirtschaftslehre]{Einführung in
  die Volkswirtschaftslehre} & 2 & Prof.\,Dr. Michael Rauscher\\
\end{tabularx}

% subsection Wirtschaftswissenschaften (end)

% section Übersicht (end)

\newpage

\section{Kommentare} % (fold)
\label{sec:Kommentare}

\subsection{Informatik} % (fold)
\label{sub:Informatik}

\subsubsection{Funktionale Programmierung} % (fold)
\label{ssub:Funktionale Programmierung}

% subsubsection subsubsection name (end)

\begin{description}
  \item[Lehrziel:] Die zukünftige Entwicklung großer Softwaresysteme setzt
    entsprechend Aufgabenspezifikation die Nutzung unterschiedlicher
    Programmierparadigmen innerhalb eines Softwaresystems voraus. Der
    Studierende soll unterscheiden lernen, wann welches Paradigma am
    zweckmäßigsten anzuwenden ist. Insbesondere durch Übung und Praktikum werden
    Fertigkeiten im funktionalen Programmieren erworben, wobei die
    Besonderheiten gegenüber anderen Paradigmen im Vordergrund stehen.
  \item[Inhalt:] Die funktionale Programmierung stellt das zweite wichtige
    Programmierparadigma neben der imperativen Programmierung dar. Es basiert
    auf dem Prinzip der Textersetzung („equational reasoning“) und behandelt
    Funktionen als „first class objects“. Das Modul behandelt die theoretischen
    Grundlagen der funktionalen Programmierung, führt in die Sprache Haskell ein
    und zeigt verschiedene praktische Anwendungen. Die im Modul integrierten
    Übungen ermöglichen konkrete Erfahrungen mit der funktionalen Programmierung
    zu machen.
\end{description}
\textsf{\textbf{Literatur:}}
\begin{itemize}\itemsep0pt
  \item Thompson, S. Haskell. The Craft of Functional Programming.
    Addison-Wesley, 1999. ISBN 978-0201342758
  \item Barendregt, H. P.: The Lambda Calculus: Its Syntax and Semantics.
    Studies in Logic and the Foundations of Mathematics, Band 103, North
    Holland, 1984
  \item Bird, R. Introduction to Functional Programming using Haskell.
    Addison-Wesley, 1998. ISBN 978-0134843469
  \item Bird, R.; Wadler, P.: Einführung in die funktionale Programmierung.
    Hanser Studienbücher der Informatik, Prentice-Hall, 1992
  \item Hudak, P.; Fasel, J. H.; Peterson, J.: A Gentle Introduction to
    Haskell.  http://www.haskell.org/tutorial, 1997
  \item The Hugs System 1998. http://www.haskell.org/hugs
  \item Myers, C.; Clack, Ch.; Poon, E.: Programming with Standard ML.
    Prentice-Hall, 1993
  \item Pepper, P.: Funktionale Programmierung in \textsc{Opal}, \textsc{ML},
    \textsc{Haskell} und \textsc{Gofer}. Springer Verlag, 1999
  \item Steele Jr., G. L.: Common Lisp: The Language. Digital Press, 1984
\end{itemize}

% subsection Funktionale Programmierung (end)

\subsubsection{Grundlagen der Technischen Informatik} % (fold)
\label{ssub:Grundlagen der Technischen Informatik}

\begin{description}
  \item[Lehrziel:] Teilnehmer, die dieses Modul erfolgreich absolviert haben,
    sollen in der Lage sein, Schaltnetze und Schaltwerke mit den behandelten
    Methoden unter Berücksichtigung von Optimierungszielen zu entwerfen, sowie
    gegebene Schaltungen zu analysieren und zu verstehen. Damit ist die
    Grundlage geschaffen für das Verständnis der Struktur und Funktionsweise von
    Steuerwerken und Operationswerken, die im Modul Rechnersysteme behandelt
    wird.
  \item[Inhalt:] Dieses Modul vermittelt die elementaren Grundlagen der
    digitalen Rechnertechnik
  \begin{itemize}\itemsep0pt
    \item Zahlensysteme und Zahlendarstellung
    \item Codierung
    \item Boolesche Algebra
    \item Schaltnetze (kombinatorische Schaltungen)
    \item Beschreibungsformen
    \item Minimierung von Schaltfunktionen
    \item Zeitverhalten
    \item wichtige kombinatorische Bauelemente
    \item Speicherelemente
    \item Flipflops
    \item statische und dynamische Speicherzellen
    \item Schaltwerke (sequentielle Schaltungen)
    \item Funktionsprinzip
    \item Beschreibungsformen
    \item Zeitverhalten
    \item Entwurfs- und Optimierungsmethoden
    \item Ausgewählte Aspekte des Entwurfs und der Herstellung hochintegrierter
      digitaler Schaltungen in der Praxis
  \end{itemize}
\end{description}

% subsection Grundlagen der Technischen Informatik (end)

\subsubsection{Einführung in die Programmierung mit C} % (fold)
\label{ssub:Einführung in die Programmierung mit C}

\begin{description}
  \item[Inhalt:] Dieses Modul vermittelt Grundkenntnisse zur Programmierung mit
    der Programmiersprache C.
  \begin{itemize}\itemsep0pt
    \item Begriff Informatik
    \item Zahlensysteme und elementare Logik
    \item Algorithmen (Schrittweise Verfeinerung, Pseudocode, Modularität,
      Rekursion)
    \item Syntaxbeschreibung von Programmiersprachen
    \item Struktur von C-Programmen
    \item Kontrollstrukturen in C
    \item Strukturierung von C-Programmen (Funktionen, Blöcke, Rekursion)
    \item Strukturierte Datentypen (Arrays, Strings, Strukturen, Files)
  \end{itemize}
\end{description}

% subsection Einführung in die Programmierung mit C (end)

\subsubsection{Abstrakte Datentypen} % (fold)
\label{ssub:Abstrakte Datentypen}

\begin{description}
  \item[Lehrziel:] Die Teilnehmer sollen in die Lage versetzt werden, Probleme
    als Gesamtheit von Daten und Algorithmen zu spezifizieren. Für die so
    spezifizierte Problemstellung werden von den Studierenden effiziente
    Datenstrukturen gefunden. Sie sind auch in der Lage, eine algorithmischen
    Programmiersprache zur Formulierung der Algorithmen zu nutzen. Die
    Studierenden sind befähigt, kleinere Projekte eigenständig von der Analyse
    über die Spezifikation bis zur Implementierung durchzuführen.
  \item[Inhalt:] Einführung in die Softwareentwicklung
    \begin{itemize}\itemsep0pt
      \item Strukturierte Programmierung
      \item Rekursion
      \item algebraische Spezifikation Abstrakter Datentypen
      \item Datenstrukturen zur effektiven Implementation mit Hilfe einer
        algorithmischen Sprache
      \item Spezifikation und Implementation mit unterschiedlichen
        Datenstrukturen (z.B. Liste, Keller, Schlange, Baum, Tabelle)
    \end{itemize}
\end{description}
\textsf{\textbf{Literatur:}}
\begin{itemize}\itemsep0pt
  \item Forbrig, Peter, Introduction to programming by abstract data types with
    53 examples, 59 exercises and CD-ROM, ISBN: 3446217827 (kart.) München
    [u.a.] : Fachbuchverl. Leipzig im Carl-Hanser-Verl., 2001
  \item Horebeek, Ivo van (Lewi, Johan;), Algebraic specifications in software
    engineering : an introduction, ISBN: 3540516263 ISBN: 0387516263, Berlin
    [u.a.] : Springer, 1989
  \item Horn/Kerner/Forbrig, Lehr- und Übungsbuch Informatik - Grundlagen und
    Überblick, Fachbuchverl. Leipzig im Carl-Hanser-Verl., 2003
  \item Horn/Kerner/Forbrig, Lehr- und Übungsbuch Informatik - Theorie der
    Informatik, Fachbuchverl. Leipzig im Carl-Hanser-Verl., 2002
  \item Sedgewick, Robert, Algorithmen ISBN: 3827370329 (Gb.) , München [u.a.],
    Addison-Wesley, 2003
\end{itemize}

% subsection Abstrakte Datentypen (end)

\subsubsection{Algorithmen und Datenstrukturen} % (fold)
\label{ssub:Algorithmen und Datenstrukturen}

Dringend empfohlene Voraussetzung: Vorlesung \hyperref[ssub:Abstrakte
Datentypen]{„Abstrakte Datentypen“}

\begin{description}
  \item[Lehrziel:] Selbständiges Entwickeln und Implementieren von Algorithmen,
    die Beherrschung der dazu erforderlichen Datenstrukturen und
    Entwurfsverfahren. Verbindung der Fähigkeit zur Formulierung von Verfahren
    mit Hilfe abstrakter Datentypen und der Fähigkeiten zum Programmieren in
    höheren Programmiersprachen. Fähigkeit, die Effizienz von Algorithmen,
    insbesondere ihren Zeit- und Speicherbedarf mit mathematischen Methoden zu
    analysieren und so die Qualität von verschieden Algorithmen zur Lösung von
    Problemen beurteilen zu können.

    Die Teilnehmer sollen in der Lage versetzt werden, Algorithmen zu
    spezifizieren und zu implementieren. Sie sollen auch die Komplexität
    abschätzen können und die Korrektheit einschätzen.
\end{description}
\textsf{\textbf{Inhalt:}}
\begin{itemize}\itemsep0pt
  \item Grundlegende Begriffe und formale Eigenschaften von Algorithmen
    Techniken der Algorithmenentwicklung
  \item Datentypen und Datenstrukturen
  \item Grundlegende Datenstrukturen der Informatik und ihre Implementierung
  \item Ausgewählte Algorithmen aus dem Bereich Sortieren und Suchen
  \item Asymptotische Komplexitätsanalysen
\end{itemize}
\textsf{\textbf{Literatur:}}
\begin{itemize}\itemsep0pt
  \item T. Ottmann, P. Widmayer, Algorithmen und Datenstrukturen.
  \item R. Sedgewick, Algorithmen in Java.
  \item G. Brassard, P. Bratley, Algorithmik - Theorie und Praxis.
  \item Th. Cormen, Ch. Leiserson, R. Rivest, Introduction to Algorithms.
  \item weitere aktuelle Literaturempfehlungen erfolgen zu Beginn der
    Lehrveranstaltung.
\end{itemize}

% subsection Algorithmen und Datenstrukturen (end)

\subsubsection{Einführung in die Informatik für BWLer und VWLer} % (fold)
\label{ssub:Einführung in die Informatik für BWLer und VWLer}

\begin{description}
  \item[Lehrziel:] Dieses Modul bietet eine Einführung in die Grundlagen des
    Fachgebietes Wirtschaftsinformatik. Die Schüler erhalten einen Überblick
    über Werkzeuge, Vorgehensweisen und Probleme an der Schnittstelle zwischen
    Wirtschaft und Informatik. Sie erwerben grundlegende Fähigkeiten, um
    Anwendungen mit Hilfe von Methoden der Informatik zu lösen.
\end{description}
\textsf{\textbf{Inhalt:}}
\begin{itemize}\itemsep0pt
  \item Grundlagen: Informationsverarbeitung, Rechner
  \item Entwicklung von Informationssystemen
  \item Geschäftsprozessmodellierung, Workflows und Datenmodellierung
  \item Überblick: Datenbanken und Informationssysteme
  \item Auswertung von Daten: Datenanalyse, Mining, Data Warehouses
  \item Datensicherheit und Datenschutz
  \item Computernetze, verteilte Systeme
\end{itemize}
\textsf{\textbf{Literatur:}}
\begin{itemize}\itemsep0pt
  \item Hans Robert Hansen, Gustaf Neumann: Wirtschaftsinformatik I, UTB 2005
  \item (Hans Robert Hansen, Gustaf Neumann: Wirtschaftsinformatik II, UTB
    2005), optional
  \item Helmut Herold / Bruno Lurz / Jürgen Wohlrab: Grundlagen der Informatik,
    Pearson Studium, 2006
\end{itemize}

% subsection Einführung in die Informatik für BWLer und VWLer (end)

\subsubsection{Einführung in die Praktische Informatik} % (fold)
\label{ssub:Einführung in die Praktische Informatik}

\begin{description}
  \item[Inhalt:] Sicherer Umgang bei der Erstellung von Programmen kleinerer und
    mittlerer Größe zur Lösung von Aufgaben aus den Ingenieurs- und
    Geisteswissenschaften. Darüber hinaus werden die Grundzüge über das
    Zusammenspiel von Hardware und Betriebssystem sowie Betriebssystem und
    Programm vermittelt. Ein weiteres Ziel ist, dass die Absolventen in der Lage
    sind, hardwarenah zu programmieren. Ferner sind die Teilnehmer in der Lage,
    für gegebene technische Problemstellungen entsprechende Algorithmen zu
    entwickeln.

    Dieses Modul ist eine Fortführung des Moduls Informatik I. Aufbauend auf
    die dort bereits vermittelten Grundkenntnisse, werden in diesem Modul
    komplexere Algorithmen, dynamische Datenstrukturen sowie einfache
    Komplexitätsabschätzungen behandelt.

    \begin{itemize}\itemsep0pt
      \item Grundlegende Algorithmen aus den Bereichen Filtern, Konvertieren und
        Suchen
      \item Entwurf und Realisierung von Textbasierten
        Ein-/Ausgabeschnittstellen
      \item Aufwandsabschätzungen ($\mathcal{O}$-Kalkül)
      \item Variablen, Adressen und Parameterübergabemechanismen
      \item Dynamische Datenstrukturen, Listen, Bäume, Hashtabllen und
        Komplexität dieser Datenstrukturen
      \item Verteilte Verwaltung und Bearbeitung von C-Programmen
      \item Programmieren einfacher Scanner- und Parserfunktionalitäten
      \item Bearbeiten einer größeren Programmieraufgabe.
    \end{itemize}
  \item[Voraussetzung:] Kenntnisse der Programmiersprache C
\end{description}

% subsection Einführung in die Praktische Informatik (end)

\subsubsection{Logik} % (fold)
\label{ssub:Logik}

\begin{description}
  \item[Lehrziel:] Das Modul führt in die Grundlagen der Aussagenlogik und der
    Prädikatenlogik ein. Es schult die Fähigkeit zur präzisen Definition von
    Begriffen und zur folgerichtigen Argumentation. Es demonstriert die
    Beziehung zwischen Syntax und Semantik am Beispiel von Logikkalkülen.
\end{description}
\textsf{\textbf{Inhalt:}}
\begin{itemize}\itemsep0pt
  \item Explizite Definitionen und direkte Beweise am Beispiel
    mengentheoretischer Grundkonzepte
  \item Induktive Definitionen und Beweise am Beispiel einfacher formaler
    Sprachen
  \item Syntax und Semantik der Aussagenlogik
  \item Normalformen, Erfüllbarkeit, Hornformeln, Markierungsalgorithmus
  \item aussagenlogische Resolution
  \item Folgern und Ableiten
  \item Syntax und Semantik der Prädikatenlogik erster Stufe
  \item Erfüllbarkeit, Modelle, Allgemeingültigkeit
  \item bereinigte Pränexform, Skolemform, Herbrandstrukturen
  \item Unifikation und prädikatenlogische Resolution
  \item Ausblick auf weitere informatikrelevante Logikkalküle, z.B. modale und
    temporale Logiken.
\end{itemize}

% subsection Logik (end)

% subsection Informatik (end)

\subsection{Chemie/Medizin} % (fold)
\label{sub:Chemie/Medizin}

\subsubsection{Medizinische Neurobiologie} % (fold)
\label{ssub:Medizinische Neurobiologie}

\begin{description}
  \item[Allgemeiner Inhalt der Vorlesung:] Aufbau der Zellen und mikroskopische
    Anatomie des Nervensystems, makroskopische Anatomie und Funktion des
    Nervensystems, Grundlagen der mikroskopischen und makroskopischen Anatomie
    der Sinnesorgane , Entwicklungs- und experimentelle Neurobiologie
  \item[Ziel:] Erreichen fachübergreifender Kompetenz zum Verständnis
    physiologischer, pathophysiologischer und neurologischer Unterrichtsinhalte
\end{description}
\textsf{\textbf{Einheit:}}
\begin{enumerate}\itemsep0pt
  \item Nervensystem -- allgemeiner Überblick
  \begin{itemize}\itemsep0pt
    \item Nervensystem als Verband neuronaler und glialer Zellen
    \item Grundprinzip einer neuronalen Organisation
  \end{itemize}
  \item Zelluläre Neurobiologie
  \item Makroskopischer Aufbau
  \begin{itemize}\itemsep0pt
    \item zentrales NS / peripheres NS / enterisches NS
    \item animalisches / vegetatives (sympathisches / parasympathisches) NS
  \end{itemize}
  \item Sinnesorgane
  \item Funktionelle Neuroanatomie
  \item Entwicklung
  \item Experimentelle Neurobiologie
\end{enumerate}

% subsection Medizinische Neurobiologie (end)

\subsubsection{Anatomie des Nervensystems} % (fold)
\label{ssub:Anatomie des Nervensystems}

\begin{description}
  \item[Lehrziel:] Bau und Verknüpfung des zentralen Nervensystems des Menschen
  \item[Inhalt:] weitgehend entsprechend Gegenstandskatalog für den
    schriftlichen Teil des Ersten Abschnitts der Ärztlichen Prüfung (ÄAppO 2002
    IMPP-GK 1), klinische Beispiele
\end{description}
\textsf{\textbf{Literatur:}}\\
Lehrbücher:
\begin{itemize}\itemsep0pt
  \item Schiebler: Anatomie
  \item Zilles-Rehkämper: Funktionelle Neuroanatomie
  \item Trepel: Neuroanatomie
  \item Benninghoff: Anatomie
\end{itemize}
Atlanten:
\begin{itemize}\itemsep0pt
  \item Sobotta: Anatomie
  \item Tillmann: Atlas der Anatomie
  \item Köpf-Maier: Atlas der Anatomie des Menschen
\end{itemize}

% subsection Anatomie des Nervensystems (end)

\subsubsection{Anatomie der Sinnesorgane} % (fold)
\label{ssub:Anatomie der Sinnesorgane}

\begin{description}
  \item[Lehrziel:] Topographie, Bau und Funktion der Sinnesorgane
  \item[Inhalt:] weitgehend entsprechend Gegenstandskatalog für den
    schriftlichen Teil des Ersten Abschnitts der Ärztlichen Prüfung (ÄAppO 2002
    IMPP-GK 1), klinische Bezüge
\end{description}
\textsf{\textbf{Literatur:}}
\begin{itemize}\itemsep0pt
  \item Schiebler: Anatomie
  \item Junqueira: Histologie
  \item Zilles Rehkämper: Funktionelle Neuroanatomie
  \item Kahle/Frotscher: Neuroanatomie und Sinnesorgane
  \item Stevens Lowe: Histologie des Menschen
  \item Sobotta/Welsch: Histologie
\end{itemize}

% subsection Anatomie der Sinnesorgane (end)

\subsubsection{Grundlagen Chemie für Mediziner} % (fold)
\label{ssub:Grundlagen Chemie für Mediziner}

\textsf{\textbf{Inhalt:}}
\begin{itemize}\itemsep0pt
  \item grundlegende chemische Gesetze, Molbegriff
  \item Atombau und Periodensystem der Elemente
  \item Arten der chemischen Bindung
  \item Stoffe und ihre Zustandsformen, Gasgesetze
  \item Lösungen und Gehaltsgrößen, Kolloide, Osmose, Verteilungsgleichgewichte
  \item Energetik und Kinetik chemischer Reaktionen
  \item Elektrolyte, Säure-Base-Reaktionen, pH-Wert, Puffer
  \item Redoxreaktionen und elektrochemische Reaktionen
  \item Chemie der Hauptgruppenelemente und einiger ausgewählter
    Nebengruppenelemente
  \item Kohlenwasserstoffe, Halogenkohlenwasserstoffe, Alkanole, Alkanale,
    Alkansäuren, Ether, Amine, Aminosäuren
  \item Peptide und Proteine, Kohlenhydrate, Fette in einer Übersicht
  \item Isomerie und ihre verschiedenen Erscheinungsformen
\end{itemize}
\textsf{\textbf{Literatur:}}
\begin{itemize}\itemsep0pt
  \item Henning, H., Sicker, D., Franz, J.: Grundlagen der Chemie für Mediziner
    systematisch (auch für Studierende anderer biologisch orientierter
    Wissenschaften), 1. Auflage 2002, UNI-MED Verlag AG, Bremen. Bisherige
    Auflagen erschienen im Johann Ambrosius Barth-Verlag
  \item Zeeck, A.: Chemie für Mediziner, 5. Auflage 2003, Verlag Urban \&
    Fischer, München
  \item Riedel, E.: Anorganische Chemie, 5. Auflage 2002, Verlag Walter de
    Gruyter, Berlin
  \item Wünsch, K.-H., Miethchen, R., Ehlers, D.: Grundkurs Organische Chemie,
    6. Auflage 1993, Johann Ambrosius Barth-Verlag
  \item Fittkau, S.: Organische Chemie, 6. Auflage 1988, Fischer-Verlag, Jena
\end{itemize}

% subsection Grundlagen Chemie für Mediziner (end)

\subsubsection{Technische Chemie} % (fold)
\label{ssub:Technische Chemie}

\begin{description}
  \item[Lehrziel:] Grundlegendes Verständnis zu Grundoperationen und
    Reaktionstechnik, Prozesssynthese

    Vertiefung ausgewählter Aspekte und Prozesskunde
    Umgang mit komplexen technisch-chemischen Apparaten

    Auswertung /Diskussion / Fehlerbetrachtung.
  \item[Grundlagen:] Mechanische Grundoperationen; thermische Trennverfahren;
    ideale und reale Reaktoren; Verweilzeitverteilung; Berechnung von Reaktoren
  \item[Vertiefung:] Mehrphasensysteme, Wärmeeffekte, spezielle Trennverfahren,
    Prozesskunde
  \item[Praktikum:] Extraktion, Adsorption, Rektifikation, Wärmetauscher,
    Filtration, Verweilzeitverteilung, begaster Rührkessel, Mikroreaktor,
    Elektrochemie, Modellierung
\end{description}

% subsubsection Technische Chemie (end)

\subsubsection{Anatomie für Nichtmediziner -- Teil I} % (fold)
\label{ssub:Anatomie für Nichtmediziner -- Teil I}

\begin{description}
  \item[Inhalt:] Bau der Zellen, Gewebe und Organe des Menschen; mikroskopische
    und makroskopische Anatomie wichtiger Systeme des Menschen
  \item[Literatur:] Schiebler: Anatomie; Faller (bearbeitet von M. Schünke): Der
    Körper des Menschen -- Einführung in Bau und Funktion
\end{description}

% subsubsection Anatomie für Nichtmediziner -- Teil I (end)

\subsubsection{Allgemeine Zytologie und Histologie} % (fold)
\label{ssub:Allg. Zytologie und Histologie}

\begin{description}
  \item[Inhalt:] Bau und Funktionen der Zellen und der Gewebe des Menschen
    weitgehend entsprechend dem Gegenstandskatalog für den schriftlichen Teil
    des Ersten Abschnitts der Ärztlichen Prüfung (ÄAppO 2002 IMPP-GK 1)
\end{description}
\textsf{\textbf{Literatur:}}\\
Lehrbücher:
\begin{itemize}\itemsep0pt
  \item Schiebler: Anatomie
  \item Benninghoff: Anatomie
  \item Lüllmann-Rauch: Histologie
  \item Welsch: Lehrbuch Histologie
  \item Ulfig: Histologie
\end{itemize}
Atlanten:
\begin{itemize}\itemsep0pt
  \item Kühnel: Taschenatlas
\end{itemize}

% subsubsection Allgemeine Zytologie und Histologie (end)

% subsection Chemie/Medizin (end)

\subsection{Biologie} % (fold)
\label{sub:Biologie}

\subsubsection{Grundlagen der Genetik} % (fold)
\label{ssub:Grundlagen der Genetik}

\begin{description}
  \item[Inhalt:] Einführung in die klassische und molekulare Genetik. Es werden
    die Grundlagen über die Erhaltung und Weitergabe der genetischen Information
    vermittelt. DNA, RNA, Chromosomen. Mutationen, Mechanismen der
    Rekombination, Einführung in Humangenetik, Einführung in die Gentechnik
    u.a..

    Prüfungsfach im Vordiplom:\\
    Grundpraktikum Genetik. Kreuzungsgenetik (Drosophila);
    Chromosomendarstellungen Mensch (mit Chromosomenaberrationen), Pflanzen,
    Drosophila; Mikrobengenetik (Mutationen, Reparaturmechanismen,
    Bakteriophagenkreuzung)
\end{description}
\textsf{\textbf{Literatur:}}
\begin{itemize}\itemsep0pt
  \item Hennig, Genetik, Springer Verlag 2002, 3. Auflage
  \item Knippers, Molekulare Genetik, Thieme Verlag 2001, 8. Auflage
  \item Munk (Hrsg.), Grundstudium Biologie/Genetik, Spektrum Verlag 2001
  \item Alberts/Johnson/Lewis/Raff/Roberts/Walter, Molekularbiologie der Zelle,
    Wiley VCH, 2004, 4. Auflage
\end{itemize}

% subsubsection Grundlagen der Genetik (end)

\subsubsection{Evolution} % (fold)
\label{ssub:Evolution}

\begin{description}
  \item[Inhalt:] Die Vorlesung gibt den Stand der biologischen Evolutionslehre
    wieder und erklärt ihren Aussagewert sowie die Methoden der
    Erkenntnisgewinnung auf dem Gebiet der naturwissenschaftlichen
    Evolutionslehre.
  \begin{enumerate}\itemsep0pt
    \item Die Entwicklung des Evolutionsgedankens (Geschichte der Evolutionslehre)
    \item Die klassischen und die molekularen Fachgebiete der Biologie als
      „Säulen“ der Evolutionslehre
    \item Die „Triebkräfte“ oder Mechanismen der Evolution (Mutation,
      Rekombination, Isolation, Selektion und Gendrift)
    \item Die Entstehung des Lebens von der „Ursuppe“ bis zu den vielzelligen
      Organismen
    \item Denkweisen in der naturwissenschaftlichen Evolutionsbiologie (Hierbei
      wird auch dargelegt, warum Kre"-a"-tio"-nis"-mus- und Intelligent Design-Lehren
      nicht Gegenstand naturwissenschaftlicher Betrachtungen sein können).
  \end{enumerate}
  \item[Literatur:] \url{http://www.vifabio.de/darwinjahr2009/}
\end{description}

% subsubsection Evolution (end)

% subsection Biologie (end)

\subsection{Anglistik} % (fold)
\label{sub:Anglistik}

\subsubsection{7th Fulbright Lecture Series: American Culture, Past and Present}
\label{ssub:7th Fulbright Lecture Series: American Culture, Past and Present}

This lecture series brings American Fulbright professors to Rostock, who will be
teaching at German universities in the summer of 2009. The guest speakers will
lecture on their particular field within American Studies, addressing such
issues as history, politics, literature, ethnicity, and popular culture. The
final list of speakers and topics will be published at the beginning of the
summer term. Each lecture will consist of a brief introduction of the speaker,
his/her presentation (in English) and question time (in English or German).

% subsubsection 7th Fulbright Lecture Series: American Culture, Past and Present (end)

% subsection Anglistik (end)

\subsection{Geschichte} % (fold)
\label{sub:Geschichte}

\subsubsection{Epochen im Überblick: Von der Reformation zum Westfälischen
Frieden} % (fold)
\label{ssub:Epochen im Überblick: Von der Reformation zum Westfälischen Frieden}

\begin{description}
  \item[Inhalt:] Die Vorlesung gibt einen Überblick über die Strukturgeschichte
    von der Reformation bis zum Westfälischen Frieden (1500--1648) in zwei
    Semestern, aber jedes Semester bildet in sich eine thematische Einheit und
    kann für sich besucht werden. Im Wintersemester 2009/2010 werden folgende
    Themenbereiche behandelt: Bevölkerung, Wirtschaft, Agrarverfassung,
    Reformation, im Semester danach: Kirche und Staat, Politische Geschichte,
    Militärische Rvolution, 30jähriger Krieg Verfasssungskonflikte. Ausblicke in
    die Kulturgeschichte werden gegeben.
\end{description}
\textsf{\textbf{Literatur:}}
\begin{itemize}\itemsep0pt
  \item Aubin, Hermann und Zorn, Wolfgang (Hrsg.): Handbuch der deutschen
    Wirtschafts- und Sozialgeschichte. Band 1. Stuttgart 1971.
  \item Burkhardt, Johannes: Der Dreißigjährige Krieg. Frankfurt am Main 1992
    (Moderne Deutsche Geschichte 2).
  \item Cipolla, Carlo M. und Borchardt, Knut (Hrsg.): Europäische
    Wirtschaftsgeschichte. Band 2: Das 16. und 17. Jahrhundert. Stuttgart 1983
    (UTB 1268).
  \item Dülmen, Richard van: Entstehung des frühneuzeitlichen Europa
    1150-1648.  Frankfurt 1982.
  \item Fischer, Wolfram, van Houte, Jan A., Kellenbenz, Hermann, Mieck, Ilja,
    Friedrich Vittinghoff (Hrsg.): Handbuch der europäischen Wirtschafts- und
    Sozialgeschichte. Band 3: Europäische Wirtschafts- und Sozialgeschichte
    vom ausgehenden Mittelalter bis zur Mitte des 17. Jahrhunderts. Hrsg. von
    Hermann Kellenbenz. Stuttgart 1986.
  \item Gebhardt, Bruno: Handbuch der deutschen Geschichte. 9. Auflage hrsg.
    v. Herbert Grundmann. Band 2. Stuttgart 1970. [Auch als Taschenbücher bei
    dtv]
  \item Goertz, Hans-Jürgen: Religiöse Bewegungen in der Frühen Neuzeit.
    München 1993 (Enzyklopädie Deutscher Geschichte 20).
  \item Hippel, Wolfgang von: Armut, Unterschichten, Randgruppen in der frühen
    Neuzeit.  München 1995 (Enzyklopädie Deutscher Geschichte 34).
  \item Holenstein, André: Bauern zwischen Bauernkrieg und Dreißigjährigem
    Krieg.  München 1996 (Enzyklopädie Deutscher Geschichte 38).
  \item Schulze, Winfried: Deutsche Geschichte im 16. Jahrhundert 1500-1618.
    Frankfurt am Main 1987 (Moderne Deutsche Geschichte 1)
  \item Vogler, Günter: Europas Aufbruch in die Neuzeit. Stuttgart 2003
    (Handbuch der Geschichte Europas Band 5).
\end{itemize}
Eine ausführlichere Literaturliste wird in StudIP bereitgestellt.

% subsubsection Epochen im Überblick: Von der Reformation zum Westfälischen
% Frieden (end)

\subsubsection{Epochen im Überblick: Von der Reformation zum Westfälischen
Frieden (Teil 2)} % (fold)
\label{ssub:Epochen im Überblick: Von der Reformation zum Westfälischen Frieden
(Teil 2)}

\begin{description}
  \item[Inhalt:] Die Vorlesung setzt den Überblick über die Strukturgeschichte
    fort, der im ersten Teil die Zeit von der Reformation bis zum Westfälischen
    Frieden (1500--1648), im zweiten den europäischen Absolutismus (1648--1789)
    zum Gegenstand hat. Jeder Teil beansprucht zwei Semester, aber jedes
    Semester bildet in sich eine thematische Einheit und kann für sich besucht
    werden. Im Sommersemester 2008 kommt der erste Teil zum Abschluss; es werden
    folgende Themenbereiche behandelt: Kirche und Staat, Politische Geschichte,
    Militärverfassung, 30jähriger Krieg. Ausblicke in die Kulturgeschichte
    werden gegeben. Arbeitsmaterial ist im Printzentrum im Rostocker Hof
    (Galerie) erhältlich. Die Teilnahme an der Vorlesung ist nur sinnvoll, wenn
    alle Teilnehmer das Arbeitsmaterial in jeder Sitzung dabei haben.
\end{description}

% subsubsection Epochen im Überblick: Von der Reformation zum Westfälischen
% Frieden (Teil 2) (end)

\subsubsection{Europäischer Absolutismus 1648--1789. Teil 1} % (fold)
\label{ssub:Europäischer Absolutismus 1648--1789. Teil 1}

\begin{description}
  \item[Inhalt:] Diese strukturgeschichtliche Vorlesung ist auf zwei Semester
    angelegt; sie soll einen Überblick vermitteln. Als thematische Schwerpunkte
    sind vorgesehen: politische Theorie und Verfassung nach den konkurrierenden
    herrschaftlichen und genossenschaftlichen Prinzipien. Dabei stehen Nord-,
    Mittel- und Westeuropa im Vordergrund. Es folgen -- voraussichtlich in einem
    späteren Semester -- die Bereiche der Wirtschaft, der Gesellschaft und der
    Politik. Abschließend werden die Reformen des Aufgeklärten Absolutismus
    dargestellt und in ihrer Bedeutung für die Moderne erörtert. Die in der
    Vorlesung verwendeten Quellentexte liegen als Kopiervorlagen im Copyshop im
    Brunnenhof bereit. Es ist unerlässlich, dass alle Teilnehmerinnen und
    Teilnehmer diese Quellen in jede Sitzung mitbringen.
\end{description}
\textsf{\textbf{Literatur:}}
\begin{itemize}\itemsep0pt
  \item Asch, Ronald (Hrsg.): Der Absolutismus - ein Mythos? Köln u. a. 1996.
  \item Brandt, Peter: Von der Adelsmonarchie zur königlichen "Eingewalt", in: HZ 250, 1990, S. 33-72.
  \item Barudio, Günter: Das Zeitalter des Absolutismus und der Aufklärung
    1648-1779. Fischer Weltgeschichte 25. Frankfurt am Main 1981.
  \item Duchhardt, Heinz: Das Zeitalter des Absolutismus. München 3. Aufl. 1998.
  \item Hinrichs, Ernst (Hrsg.): Absolutismus. Frankfurt am Main 1987.
  \item Hinrichs, Ernst: Fürsten und Mächte. Zum Problem des europäischen Absolutismus. Göttingen 2000.
  \item Kunisch, Johannes: Absolutismus. Europäische Geschichte vom
    Westfälischen Frieden bis zur Krise des Ancien Regime Göttingen 2. Aufl.
    1999.
  \item Wehler, Hans-Ulrich: Deutsche Gesellschaftsgeschichte 1700-1815. München 2.  Aufl. 1989.
\end{itemize}
Eine ausführliche Literaturliste steht in der Webseite des Historischen
Instituts, Lehre, Materialien, Literatur zu Lehrveranstaltungen

% subsubsection Europäischer Absolutismus 1648--1789. Teil 1 (end)

\subsubsection{Europäischer Absolutismus 1648--1789. Teil 2} % (fold)
\label{ssub:Europäischer Absolutismus 1648--1789. Teil 2}

Dies ist die Fortsetzung der Veranstaltung „\hyperref[ssub:Europäischer
Absolutismus 1648--1789. Teil 1]{Europäischer Absolutismus 1648--1789. Teil 1}“.
Weitere Informationen sind dort zu finden.

% subsubsection Europäischer Absolutismus 1648--1789. Teil 2 (end)

\subsubsection{Diaspora} % (fold)
\label{ssub:Diaspora}

\begin{description}
  \item[Inhalt:] Was ist das überhaupt -- Diaspora? Und warum scheint der
    Begriff Konjunktur zu haben? Was heißt es, wenn von „Opfer-Diaspora“,
    „Arbeits-Diaspora“, „Handels-Diaspora“ oder „imperialer Diaspora“ gesprochen
    wird? Was, wenn Diaspora als „diskontinuierlicher sozialer Raum“ (Jürgen
    Osterhammel) oder als „Paradigma der globalisierten Welt“ (Ruth Mayer)
    erscheint? Was schließlich, wenn Historikerinnen und Historiker Diaspora im
    Rahmen von Konzepten wie „Kulturelles Gedächtnis“, „Erfundene Traditionen“,
    „Black Atlantic“ oder „Hybridität“ in den Blick nehmen. Ausgehend von
    Begriffsklärungen und Typologisierungsangeboten, versucht die Vorlesung,
    diese und andere Fragen zu beantworten, indem sie ihren Gegenstand als
    Chance versteht, Geschichte transnational, transkontinental und
    transkulturell zu betreiben. 
  \item[Literatur:] Robin Cohen, Global Diasporas. An Introduction, 1997
\end{description}

% subsubsection Diaspora (end)

\subsubsection{Die Stadt der Moderne (16.-20. Jahrhundert)} % (fold)
\label{ssub:Die Stadt der Moderne (16.--20. Jahrhundert)}

\begin{description}
  \item[Inhalt:] Die Vorlesung wird die Stadtentwicklung der Neuzeit
    vergleichend untersuchen. Dabei sollen Idealstadtentwürfe der frühen Neuzeit
    wie der Moderne besondere Berücksichtigung finden. Einleitend wird
    theoretisch der Stadtbegriff erörtert. Danach folgen ausgewählte empirische
    Fallbeispiele. Diese reichen vom 16. Jahrhundert bis in die Gegenwart. Der
    Ablaufplan wird in StudIP bereitgestellt.
\end{description}
\textsf{\textbf{Literatur:}}
\begin{itemize}\itemsep0pt
  \item Albers, Gerd: Stadtplanung. Eine illustrierte Einführung. Darmstadt
    2008.
  \item Beier, Rosemarie (Hrsg.): Aufbau West Aufbau Ost. Berlin 1997.
  \item Beneovolo, Leonardo: Geschichte der Architektur des 19. und 20.
    Jahrhunderts. 3 Bände.  3. Auflage München 1994.
  \item Düwel, Jörn: Baukunst voran. Architektur und Städtebau in der SBZ/DDR.
    Berlin 1995.
  \item Friedrichs, Jürgen: Stadtanalyse.  Soziale und räumliche Organisation
    der Gesellschaft. 3. Auflage Opladen 1983.
  \item Gerteis, Klaus: Die deutschen Städte in der frühen Neuzeit. Zur
    Vorgeschichte der „bürgerlichen Welt“. Darmstadt 1986.
  \item Mumford, Lewis: Die Stadt. Geschichte und Ausblick. München 1979 (dtv
    Wissenschaft 4326).
  \item Reulecke, Jürgen (Hg.): Die deutsche Stadt im Industriezeitalter.
    Beiträge zur modernen deutschen Stadtgeschichte. 2. Auflage Wuppertal 1980.
  \item Schilling, Heinz: Die Stadt in der frühen Neuzeit. München 1993
    (Enzyklopädie deutscher Geschichte 24).
\end{itemize}
Eine ausführlichere Bibliografie wird in StudIP bereitgestellt.

% subsubsection Die Stadt der Moderne (16.--20. Jahrhundert) (end)

\subsubsection{Geschichte des deutschen Kommunismus (Teil 1: 1918--1945)}
% (fold)
\label{ssub:Geschichte des deutschen Kommunismus (Teil 1: 1918--1945)}

\begin{description}
  \item[Inhalt:] Der deutsche Kommunismus nach dem Ersten Weltkrieg
    repräsentierte dreierlei: den radikalen Flügel der sozialen Bewegung, die
    Partei, die sich dem sowjetischen Herrschafts- und Gesellschaftsmodell
    verschrieben hatte und zuletzt die Trägerin einer Ideologie, die ein
    absolutes Wahrheitsmonopol beanspruchte. Seine Entwicklung war zugleich
    reich an Brüchen, Wendungen und inneren Widersprüchen. Erst mit der
    „Stalinisierung“ der KPD seit Mitte der zwanziger Jahre trat eine äußerlich
    sichtbare Homogenisierung ein, die zugleich die Partei zu einem
    gesellschaftlichen Außenseiter-Dasein verurteilte, die jedoch wiederum
    Abspaltungen und Richtungsstreitigkeiten verursachte. Nach 1933 veränderten
    sich mit Verfolgung, Untergrund und Widerstand auf der einen sowie Exil auf
    der anderen Seite Programme, Konzeptionen und Tätigkeitsfelder erneut. Diese
    vielschichtige Entwicklung wird mit der Geschichte der KPD im Mittelpunkt
    nachgezeichnet, die kommunistischen „Splitterparteien“ werden mit
    einbezogen.
\end{description}
\textsf{\textbf{Literatur:}}
\begin{itemize}\itemsep0pt
  \item Gerd Koenen: Was war der Kommunismus? Göttingen 2010
  \item Kommunistische Bewegung und realsozialistischer Staat. Beiträge zum
    deutschen und internationalen Kommunismus von Hermann Weber. Ausgewählt,
    herausgegeben und eingeleitet von Werner Müller, Köln 1988
  \item Jerzy Holzer: Der Kommunismus in Europa. Poltische Bewegung und
    Herrschaftssystem, Frankfurt am Main 1988
\end{itemize}

% subsubsection Geschichte des deutschen Kommunismus (Teil 1: 1918--1945) (end)

% subsection Geschichte (end)

\subsection{Theologie} % (fold)
\label{sub:Theologie}

\subsubsection{Einführung in die Geschichte des Christentums} % (fold)
\label{ssub:Einführung in die Geschichte des Christentums}

\begin{description}
  \item[Inhalt:] Der Grundkurs, in dem sich Elemente der Vorlesung und des
    Kolloquiums verbinden, richtet sich vor allem an Studierende für das Lehramt
    im Grundstudium. Er bietet eine Einführung in das Studium der
    Kirchengeschichte. Dies geschieht an ausgewählten Themenbereichen, in denen
    sich das Spektrum der zweitausendjährigen Kirchengeschichte widerspiegelt.
    In jeder Sitzung wird das jeweilige Thema anhand eines ausgewählten Textes
    vertiefend behandelt.
  \item[Literatur:] Zur Anschaffung wird empfohlen: B. Moeller, Geschichte des
    Christentums in Grundzügen (UTB 905), Göttingen: Vandenhoeck \& Ruprecht, 8.
    Aufl. 2004
\end{description}

% subsubsection Einführung in die Geschichte des Christentums (end)

\subsubsection{Bibelkunde des Neuen Testaments} % (fold)
\label{ssub:Bibelkunde des Neuen Testaments}

\begin{description}
  \item[Inhalt:] Die Übung bietet einen Überblick über den Aufbau der
    neutestamentlichen Schriften sowie kurze Einführungen in deren
    Entstehungssituation und theologisches Profil. Von den Teilnehmerinnen und
    Teilnehmern wird erwartet, dass sie parallel zur Übung das Neue Testament
    lesen und die jeweils zu behandelnden Schriften unter vorgegebenen
    Fragestellungen durcharbeiten.
  \item[Literatur:] K.-M. Bull, Bibelkunde des Neuen Testaments, 5. Aufl.,
    Neukirchen-Vluyn 2006.
\end{description}

% subsubsection Bibelkunde des Neuen Testaments (end)

\subsubsection{Geschichte der frühjüdischen Literatur} % (fold)
\label{ssub:Geschichte der frühjüdischen Literatur}

\begin{description}
  \item[Inhalt:] Die Autoren der neutestamentlichen Schriften nehmen in
    vielfältiger Weise auf die Diskurse des hellenistischen Judentums Bezug. Die
    Vorlesung bietet einen Überblick über die jüdische Literatur aus der Zeit 2.
    Jhd. v. Chr. bis 1. Jhd. n. Chr., um auf diese Weise einen ersten Zugang zu
    dieser geistigen Welt zu ermöglichen.
  \item[Literatur:] Jüdische Schriften aus hellenistisch-römischer Zeit (JSHRZ),
    hg. v. W.G. Kümmel u.a., Gütersloh 1973ff (Übersetzungen)

    Neues Testament und Antike Kultur (Hrsg. K. Erlemann u.a.), Band 1,
    Abschnitt 1.3.2 (dort weitere Literatur)
\end{description}

% subsubsection Geschichte der frühjüdischen Literatur (end)

% subsection Theologie (end)

\subsection{Germanistik/Kommunikation} % (fold)
\label{sub:Germanistik/Kommunikation}

\subsubsection{„Mach mal den Flyer! Du hast das doch studiert!“
Kommunikationswissenschaft zwischen Klischee und Wirklichkeit} % (fold)
\label{ssub:„Mach mal den Flyer! Du hast das doch studiert!“
Kommunikationswissenschaft zwischen Klischee und Wirklichkeit}

\begin{description}
  \item[Inhalt:] Fast allen Wissenschaften sind im fachfremden öffentlichen
    Bewusstsein bestimmte klischeehafte Vorstellungen zugeordnet. Mathematiker
    können gut rechnen, Informatiker programmieren Computer, Germanisten wollen
    Deutschlehrer werden und Kommunikationswissenschaftler -- können gut
    kommunizieren? sind gute Journalisten? oder perfekt für die
    Marketingabteilung?

    Welch weites interdisziplinäres Feld man betreten hat, als man beschloss,
    „Kommunikationswissenschaft“ zu studieren, wird spätestens dann deutlich,
    wenn man sich mit Kommunikationsstudenten aus unterschiedlichen
    Universitäten über die Fachinhalte unterhält: Während sich etwa der eine mit
    linguistischen und kommunikationspsychologischen Fragestellungen
    auseinandersetzt, geht der andere z.\,B. publizistisch oder journalistisch
    an seinen Studiengang.

    Ausgehend von der Selbstverständnis-Erklärung der „Deutschen Gesellschaft
    für Publizistik und Kommunikationswissenschaft“ sowie dem Bericht des
    Wissenschaftsrates zum Stand der Medien- und Kommunikationswissenschaften in
    Deutschland zeichnet dieses Seminar die historische Entwicklung der
    verschiedenen kommunikationswissenschaftlichen Perspektiven im
    deutschsprachigen Raum nach und bettet sie in einen internationalen Kontext
    ein.

    Ziel ist es, die verfolgten Perspektiven greifbarer zu machen,
    interdisziplinäre Zusammenhänge bewusst zu machen und die faszinierende
    Spannweite des Faches aufzuzeigen.
\end{description}

% subsubsection „Mach mal den Flyer! Du hast das doch studiert!“
% Kommunikationswissenschaft zwischen Klischee und Wirklichkeit (end)

\subsubsection{Sprache -- Medien -- Kommunikation: Von den Anfängen der
deutschen Sprache zur Internet-Kommunikation} % (fold)
\label{ssub:Sprache -- Medien -- Kommunikation: Von den Anfängen der deutschen
Sprache zur Internet-Kommunikation}

\begin{description}
  \item[Inhalt:] Die Übung bietet einen Überblick über die historische
    Entwicklung menschlicher Sprache und Kommunikation, mit besonderer
    Berücksichtigung der Geschichte und Gegenwart der deutschen Sprache. Dabei
    werden formale Merkmale immer in Beziehung gesetzt zu den medialen
    Bedingungen der jeweiligen Zeit, da Sprache nie losgelöst von ihren Nutzern
    (der „Sprachgemeinschaft“) und ihrem Zweck (nämlich Kommunikation zu
    ermöglichen, um ein bestimmtes Ziel zu verfolgen) betrachtet werden sollte.

    Die Veranstaltung kann als ergänzende Veranstaltung im
    Germanistik-Grundstudium, im Fach „Sprachliche Kommunikation und
    Kommunikationsstörungen“ (Modul J) sowie als IDS angerechnet werden.
\end{description}

% subsubsection Sprache -- Medien -- Kommunikation: Von den Anfängen der
% deutschen Sprache zur Internet-Kommunikation (end)

\subsubsection{Stationen deutscher Lyrik} % (fold)
\label{ssub:Stationen deutscher Lyrik}

\settowidth{\versewidth}{Wenn ihm Tau, hell wie Licht, aus der Locke träuft,}
\poemtitle*{Die frühen Gräber}
\begin{verse}[\versewidth]
  Willkommen, o silberner Mond, \\
  Schöner, stiller Gefährt der Nacht! \\
  Du entfliehst? Eile nicht, bleib, Gedankenfreund! \\
  Sehet, er bleibt, das Gewölk wallte nur hin. \\[1em]

  Des Maies Erwachen ist nur \\
  Schöner noch, wie die Sommernacht, \\
  Wenn ihm Tau, hell wie Licht, aus der Locke träuft, \\
  Und zu dem Hügel herauf rötlich er kömmt. \\[1em]

  Ihr Edleren, ach es bewächst \\
  Eure Male schon ernstes Moos! \\
  O wie war glücklich ich, als ich noch mit euch \\
  Sahe sich röten den Tag, schimmern die Nacht. \\
\end{verse}
\attrib{Friedrich Gottlieb Klopstock, 1764}

\poemtitle*{Die frühen Gräber}
%\settowidth{\versewidth}{Heulend reißt der Wolf den Rachen. Von Krumenhänden}
\begin{verse}[\versewidth]
  Hell strahlt eurer Glieder Asche, von Faltern \\
  im Gebüsch umflügelt, von Gewittern \\
  dampfgebügelt. Wortlos geht der Mund zur Flasche. \\[1em]

  Heulend reißt der Wolf den Rachen. Von Krumenhänden \\
  aufgehügelt, psalternd im Gespräch \\
  beflügelt, heil strahlt eurer Glieder Asche.\\
\end{verse}
\attrib{Richard Pietraß, \textit{für Richard Leising und Karl Mickel}, 2000}

\begin{description}
  \item[Literatur:] Echtermeyer. Deutsche Gedichte. Von den Anfängen bis zur
    Gegenwart. Auswahl für Schulen, hg. von Elisabeth K. Paefgen zusammen mit
    Peter Geist, 19. Auflage, Berlin 2005.

    Holznagel/Kemper/Korte/Mayer/Schnell/Sorg: Geschichte der deutschen Lyrik,
    Stuttgart 2004.
\end{description}

% subsubsection Stationen deutscher Lyrik (end)

% subsection Germanistik/Kommunikation (end)

\subsection{Allgemeine Studien} % (fold)
\label{sub:Allgemeine Studien}

\subsubsection{Was Bilder (un)sichtbar macht} % (fold)
\label{ssub:Was Bilder (un)sichtbar macht}

Aus der Perspektive verschiedener Kultur- und Naturwissenschaften soll der Frage
nachgegangen werden:
\begin{enumerate}
  \item Was Bilder sichtbar machen, in ihren Disziplinen und Kontexten. Zur
    Orientierung ist dabei zu unterscheiden zwischen Bildern im Text (Metaphern
    etc.), im Kopf (Vorstellungen, Imaginationen), an der Wand (materiell,
    extern wie Gemälde) und auf dem Schirm (Computerbilder). Fraglich sind hier
    vorrangig die letzteren beiden: also Bilder, die „auf die Augen gehen“.

    Was Bilder sichtbar machen ist
    \begin{enumerate}
      \item auf der Ebene der in den Wissenschaften thematischen Phänomenen und
        Vollzügen zu bedenken,
      \item auf der Ebene der Thematisierung, also den wissenschaftlich
        erzeugten oder verwendeten Bildern.
    \end{enumerate}
    Was macht z.\,B. ein Kruzifix (welches, wo, in welchem Gebrauch etc.?), eine
    Hostie, ein Porträtfoto, ein mittelalterliches Tafelbild oder eine
    Tomographie-Bild sichtbar?

    Und was wird mit ihm sichtbar gemacht im wissenschaftlichen Kontext und
    Gebrauch?
  \item Die Reflexion und Bestimmung der so erzeugten Sichtbarkeit provoziert
    die Gegenfrage, was in dieser Visibilisierung invisibilisiert wird: Was wird
    dabei unsichtbar gemacht, warum und mit welchen Folgen? Unsichtbarmachung
    ist mehrdeutig: was wird schlicht ausgeblendet, was wird ggf. verstellt oder
    verzerrt, wovon wird abgelenkt in der Konzentration auf das Bild, und was
    bleibt jenseits des Wahrnehmungshorizontes?

    Mit der Auswahl, die ein Bild selber ist, wird bereits anderes unsichtbar
    gelassen; mit seiner Auf- oder Darstellung wird anderes verstellt und nicht
    exponiert; mit dem Gebrauch wird ausgewählt. Damit ist die „Rückseite“ des
    Bildes und seines Gebrauchs das „Ausgeschlossene“, das Invisibilisierte.
  \item Angesichts dessen, daß hier gebraucht, gemacht, ausgewählt und indirekt
    inivisibilisiert wird, daß also selektiv mit der Ressource Aufmerksamkeit
    und gezielter Sichtbarkeit umgegangen wird, ergeben sich Fragen nach Ethos
    (vorwissenschaftlich) und Ethik (wissenschaftlich) dieser Praktiken.

    Bildethik und -politik sind die Dimensionen, die eine entsprechend
    aufmerksame Reflexion auf Bilder nicht nicht beachten kann. Am Horizont der
    doppelten Leitfrage stehen daher Verantwortung oder Rechenschaft, jedenfalls
    Reflexion auf diese ethischen Fragen.
\end{enumerate}

% subsubsection Was Bilder (un)sichtbar macht (end)

\subsubsection{Leben-Licht-Materie} % (fold)
\label{ssub:Leben-Licht-Materie}

Wir leben in einer immer komplexer werdenden Welt. Neue Ideen und Technologien
basieren einerseits auf einer hohen Spezialisierung und andererseits auf der
engen Zusammenarbeit verschiedener Wissensgebiete. Bei der Rekonstruktion
biologischer Funktionen, für den Einsatz innovativer Behandlungsmethoden mit
Lasern, zur Entwicklung neuer Oberflächen, für die Erforschung neuer Methoden
zur Energiegewinnung oder zur schadstoffarmen und energiesparenden
Materialherstellung stehen Naturwissenschaftler, Ingenieure und Mediziner in
einem interdisziplinären Dialog miteinander.

Die Vorlesungsreihe gibt einen Einblick in die wichtigsten Forschungsgebiete des
Departments „Life, Light and Matter“ und ist für Studierende aller Fakultäten
geeignet.

\begin{description}
  \item[Weitere Informationen:] Department Science and Technology of Life, Light
    and Matter

    \url{http://www.physik.uni-rostock.de/cluster/llm.htm}
\end{description}

\textsf{\textbf{Vortragsliste:}}
\LTXtable{\textwidth}{tbl-llmvortragsliste.tex}

% subsubsection Leben-Licht-Materie (end)

\subsubsection{Strukturen und Symmetrien} % (fold)
\label{ssub:Strukturen und Symmetrien}

\textsf{\textbf{Vortragsliste:}}
\LTXtable{\textwidth}{tbl-susvortragsliste.tex}

% subsubsection Strukturen und Symmetrien (end)

\subsubsection{Spielend Lernen} % (fold)
\label{ssub:Spielend Lernen}

\begin{description}
  \item[Zielstellung:] Die Fähigkeit zu lernen ist für den Menschen eine
    Grundvoraussetzung des Lebens und der Anpassung an die Umwelt. Beim Lernen
    können Erfahrungen und neu gewonnene Einsichten zum veränderten Verhalten,
    Denken und Fühlen führen. Es gibt unterschiedliche Möglichkeiten zum Erwerb
    von geistigen, körperlichen, sozialen Kenntnissen und Fertigkeiten. 

    Forscher beschäftigen sich teilweise schon seit vielen Jahren mit dem weiten
    Themenkreis des „spielbasierten Lernens“. Dazu gehören beispielsweise
    Medienwissenschaftler, Pädagogen, Lernpsychologen, Sozialwissenschaftler und
    in neuerer Zeit auch Informatiker.

    Wie kann man Lernprozesse spielerisch gestalten und dabei moderne
    Technologien nutzen? Welche Möglichkeiten für „elektronisches Lernen“
    bieten die neuen digitalen Medien? Wie kann die Motivation des Lernenden
    durch spielerische Elemente auf einem hohem Niveau gehalten werden? Wie
    lassen sich spielerische und fesselnde Elemente sinnvoll in die Lehrinhalte
    einbinden?
\end{description}

\textsf{\textbf{Vortragsliste:}}
\LTXtable{\textwidth}{tbl-slvortragsliste.tex}

% subsubsection Spielend Lernen (end)

\subsubsection{„Erfolgreich Altern“} % (fold)
\label{ssub:„Erfolgreich Altern“}

\begin{description}
  \item[Inhalt:] Steigende Lebenserwartung und niedrige Geburtenraten verändern
    die Bevölkerungsstruktur tiefgreifend. Dieser demografische Wandel gilt als
    eine der großen Herausforderungen der modernen Industriegesellschaft. Zu
    seinen Konsequenzen zählen die Verlängerung der Lebensarbeitszeit, der
    Anstieg der Pflegebedürftigkeit und Finanzierungsprobleme der Renten. Eine
    Lebenserwartung von 90 Jahren und mehr stellt die bisherige Dreiteilung des
    Lebenslaufes in die Phasen der Bildung, Erwerbstätigkeit und Freizeit nach
    Renteneintritt in Frage.

    Das zentrale Ziel der interdisziplinären Forschung ist es, die
    Selbständigkeit und Selbst­bestimmtheit im Alter zu erhöhen. Im Verbund
    forschen die Wissenschaftlerinnen und Wissen­schaftler nach neuen Lösungen
    -- auf der Ebene der medizinischen Versorgungs- und Therapieformen, im
    Bereich der sozialen Strukturen und in Bezug auf technische
    Assistenz­systeme. Die Herausforderung besteht darin, die medizinischen, die
    sozial- und geistes­wissen­schaftlichen, sowie die
    ingenieurwissenschaftlichen Aspekte ganzheitlich zu untersuchen und in ihren
    Wechselwirkungen zu verstehen.

    Die Ringvorlesung gibt einen Überblick zu Ergebnissen und Erkenntnissen aus
    den beteiligten Forschungsgebieten der Profillinie „Erfolgreich Altern“.
\end{description}

% subsubsection „Erfolgreich Altern“ (end)

\subsubsection{Einführung in die Demographie I} % (fold)
\label{ssub:Einführung in die Demographie I}

\begin{description}
  \item[Inhalt:] In der Vorlesung erfolgt die Einführung in die Grundbegriffe
    der Demographie sowie die Vorstellung grundlegender Methodiken. Schwerpunkte
    bilden die drei demographischen Prozesse: Fertilität, Mortalität und
    Migration, ihre theoretischen Grundlagen und empirischen Messkonzepte.
    Bevölkerungsprognosetechniken werden vorgestellt, die Anwendung
    demographischer Modelle auf aktuelle gesellschaftspolitische Fragen
    besprochen, sowie der Bezug zur deutschen Bevölkerungsstatistik hergestellt.
\end{description}
\textsf{\textbf{Literatur:}}
\begin{itemize}\itemsep0pt
  \item Preston, S.H., Heuveline, P., Guillot, M. (2001): Demography -Measuring
    and Modeling Population Processes. Malden/USA: Blackwell Publishers Ltd.
  \item Hinde, A. (1998): Demographic Methods. London: Arnolds Publishing.
  \item Mueller, U., Nauck, B., Dieckmann, A. (2000): Handbuch der Demographie
    1: Modelle und Methoden. Berlin: Springer.
  \item Mueller, U., Nauck, B., Dieckmann, A. (2000): Handbuch der Demographie
    2: Anwendungen. Berlin: Springer.
\end{itemize}

% subsubsection Einführung in die Demographie I (end)

% subsection Allgemeine Studien (end)

\subsection{Sozialpsychologie} % (fold)
\label{sub:Sozialpsychologie}

\subsubsection{Einführung in die Sozialpsychologie für Lehramtskandidaten}
% (fold)
\label{ssub:Einführung in die Sozialpsychologie für Lehramtskandidaten}

\begin{description}
  \item[Lehrziel:] Erwerb von grundlegendem Wissen zu sozialpsychologischen
    Basiskonzepten
\end{description}
\textsf{\textbf{Inhalt:}}
\begin{itemize}\itemsep0pt
  \item Grundbegriffe: Gedächtnis, Lernen, Wert $\times$ Erwartungstheorien
  \item Soziale Wahrnehmung (auch soz. Stereotype, implizite
    Persönlichkeitstheorien, Urteilsfehler)
  \item Konstruktion der sozialen Welt (Soziale Kognition; Urteilsbildung
    und Entscheidungen; Attribution; Einstellungen; Einstellungserwerb und
    Einstellungsänderung
  \item Gruppen (Gruppenleistungen und Gruppenstrukturen; Gruppenleistungen,
    Gruppenstrukturen (inkl. Status, Führungsverhalten)
  \item Beziehungen und Emotionen (Interpersonale Kommunikation;
    Zwischenmenschliche Anziehung; Prosoziales und hilfreiches Verhalten;
    Aggression und Feindseligkeit)
  \item Kommunikation (einfache Kommunikationsmodelle, Zusammenhang mit
    Zufriedenheit
  \item Beziehungsdreieck Eltern -- Schüler -- Lehrkräfte.
\end{itemize}
\textsf{\textbf{Literatur:}}
\begin{itemize}\itemsep0pt
  \item Zimbardo, P. G. (1999). Psychologie (7. neubearb. Aufl.). Berlin:
    Springer (besonders Kapitel 15).
  \item Stroebe, W., Hewstone, M. \& Stephenson, G. M. (Hrsg.). (1997).
    Sozialpsychologie. Eine Einführung. (Dritte, erweiterte und überarb. Aufl.).
    Berlin: Springer.
\end{itemize}

% subsubsection Einführung in die Sozialpsychologie für Lehramtskandidaten (end)

\subsubsection{Einführung in die pädagogisch-psychologische Diagnostik für
Lehramtskandidaten} % (fold)
\label{ssub:Einführung in die pädagogisch-psychologische Diagnostik für
Lehramtskandidaten}

\textsf{\textbf{Inhalt:}}
\begin{itemize}\itemsep0pt
  \item Anliegen der differenziellen Psychologie und psychol. Diagnostik (inkl.
    komplexes Schulleistungsmodell)
  \item Diagnostische Qualitätsmerkmale
  \item Intelligenz und Intelligenzmessung (beispielhaft mit Selbstversuch)
  \item Verhaltensbeobachtung
  \item Einschulungsdiagnostik
  \item Urteilsfehler
  \item Schriftliche Prüfungen aus psychol. Sicht
  \item Mündliche Prüfungen (inkl. Angstmodell)
  \item Bildungsforschung am Beispiel Pisa-Studie (Schwerpunkt auf
    Forschungsmethodik, inkl. statistische Intuitionen, Anleitung zum Lesen der
    PISA-Tabellen)
\end{itemize}

% subsubsection Einführung in die pädagogisch-psychologische Diagnostik für
% Lehramtskandidaten (end)

\textsf{\textbf{Methoden:}}
\begin{itemize}\itemsep0pt
  \item Klassische Vorlesung/Präsentation durch Dozenten;
  \item Diskussion einzelner inhaltlicher Aspekte der Vorlesung (z.B. Bekämpfung
    von Vorurteilen oder Sinn von Zensurengebung und verschiedener
    Zensierungsmodelle), kleine Experimente;
  \item praktische Erfahrungen (beispielsweise Intelligenztest-Selbstversuch mit
    Auswertung, Bewertung von Aufsätzen von Grundschülern),
  \item Kontrollfragen zu den Inhalten.
\end{itemize}

% subsection Sozialpsychologie (end)

\subsection{Maschinenbau} % (fold)
\label{sub:Maschinenbau}

\subsubsection{Fertigungslehre I} % (fold)
\label{ssub:Fertigungslehre I}

\textsf{\textbf{Inhalt:}}
\begin{enumerate}\itemsep0pt
  \item Grundlagen der Fertigungstechnik
  \item Werkstoffe
  \item Qualität
  \item Urformen
  \item Umformen (Druckumformen, Zugdruckumformen)
  \item Trennen (Zerteilen)
  \item Spanen mit geometrisch bestimmten und unbestimmten Schneiden, Abtragen,
    Beschichten
  \item Fügen
  \item Technisches Management
  \item Recycling
\end{enumerate}

% subsubsection Fertigungslehre I (end)

% subsection Maschinenbau (end)

\subsection{Philosophie} % (fold)
\label{sub:Philosophie}

\subsubsection{Aristoteles und die abendländische Philosophie} % (fold)
\label{ssub:Aristoteles und die abendländische Philosophie}

\begin{description}
  \item[Inhalt:] Aristoteles fungiert in der abendländischen
    Philosophiegeschichte ebenso häufig als Autorität, auf die man sich stützt,
    wie als Gegner, den es zu überwinden gilt. Das bedeutet, dass unser
    Aristotelesbild rezeptionsgeschichtlich meist nicht sine ira et studio
    gewonnen wird. Die Vorlesung versucht, direkt zu Aristoteles selbst
    vorzustoßen und anhand ausgewählter Texte aus seinem Werk einen Überblick
    über alle Bereiche der aristotelischen Philosophie (Ethik, Politik, Logik,
    Naturphilosophie, Metaphysik) zu geben. Dabei bleibt aber die
    Rezeptionsgeschichte in Spätantike, Mittelalter und Neuzeit nicht
    unberücksichtigt. Griechischkenntnisse werden wie gewohnt nicht
    vorausgesetzt.
  \item[Literaturempfehlung:] Christian Mueller-Goldingen, Aristoteles -- Eine
    Einführung in sein philosophisches Werk, Hildesheim 2003
\end{description}

% subsubsection Aristoteles und die abendländische Philosophie (end)

% subsection Philosophie (end)

\subsection{Wirtschaftswissenschaften} % (fold)
\label{sub:Wirtschaftswissenschaften}

\subsubsection{Einführung in die Volkswirtschaftslehre} % (fold)
\label{ssub:Einführung in die Volkswirtschaftslehre}

\begin{description}
  \item[Inhalt:] Volkswirte haben eine spezielle Art, die Welt zu betrachten,
    und in dieser Vorlesung soll ein Einblick in diese Denkweise gegeben werden.
    Insbesondere geht es dabei um die Wirkungen von (nicht nur) ökonomischen
    Anreizen auf menschliches Verhalten sowie um die Rolle von Preisen und
    Märkten. Hier werden zunächst Grundbegriffe volkswirtschaftlicher
    Theoriebildung vermittelt, die dann in den späteren Semestern des Studiums
    vertieft und systematisch erarbeitet werden. Darüber hinaus geht es um Ziele
    und Wirkungen von Wirtschaftspolitik sowie um die institutionellen
    Rahmenbedingungen, unter denen Wirtschaft abläuft: die Wirtschafts- und
    Sozialordnung der Bundesrepublik Deutschland.
\end{description}

% subsubsection Einführung in die Volkswirtschaftslehre (end)

% subsection Wirtschaftswissenschaften (end)

% section Kommentare (end)

\end{document}
